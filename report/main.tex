\documentclass[book, backcover, english, nodocumentinfo]{upmethodology-document}
\usepackage[utf8]{inputenc}
\usepackage{natbib}
\usepackage{amsmath}
\usepackage{amssymb}
\usepackage{hyperref}
\usepackage{listings}
\usepackage{textcomp}
\usepackage{color}


%link option, especialy for the table of contents
\hypersetup{
    colorlinks=true,
    linkcolor=black,
    urlcolor=blue,
    linktoc=all
}

\definecolor{codecolor}{HTML}{205280}

%environment 'lstlisting' settings
\lstset{ %
language=C,
morekeywords={BaseNIntegerList, BaseNIntegerListOfList, array, Integer, Character, Integers, Characters, ListElem, integer, integers, positive, character, characters, undefined},
basicstyle=\normalsize\color{codecolor},
keywordstyle=\normalsize\color{purple},
commentstyle=\itshape\color{gray},
identifierstyle=\textbf,
stringstyle=\color{red},
numbers=left,
numberstyle=\scriptsize,
stepnumber=1,
numbersep=7pt,
backgroundcolor=\color{yellow!3},
showspaces=false,
showstringspaces=false,
showtabs=false,
frame=single,
tabsize=4,
captionpos=b,
breaklines=true,
breakatwhitespace=true,
emph={if, then, else, endif, for, foreach, while, do, done, switch, case, break, end, procedure, function, Begin, Done},
emphstyle={\color{orange}\bfseries},
belowskip=1em,
aboveskip=1em
}

\newcommand{\nxtalgo}{\centerline{$$$$\#\#\#\#\#\#\#\#\#\#\#\#\#\#\#$$$$}}
\newcommand{\ov}{\textbf{Overview}}

\setfrontcover{classic}

\declaredocument{LO27 report}{Manipulation of lists of Base-N integers, implementation of a Radix Sort derivate}{LO27-F2015}
\setpublisher{Université de Technologie de Belfort-Montbéliard}

\incversion{\makedate{21}{12}{2015}}{Initial version.}{\upmpublic}

\addauthorvalidator*[lucas.lazare@utbm.fr]{Lucas}{LAZARE}{Author}
\addauthorvalidator*[maxime.pinard@utbm.fr]{Maxime}{PINARD}{Author}

\setdockeywords{\LaTeX, LO27, Base-N integer list, Radix sort}

\setcopyrighter{Lucas LAZARE \& Maxime PINARD}
\setprintingaddress{France}


\begin{document}

\chapter*{Introduction}
In this document, we will present you BaseNIntegerList, and BaseNIntegerListOfList libraries. We'll also make a quick introduction to the io library.\\
\qquad \textit{BaseNIntegerList} is a library implementing a doubly link list and functions to manage them.\\
\qquad \textit{BaseNIntegerListOfList} is a library implementing methods for lists of BaseNIntegerList and their management.\\
\qquad \textit{io} is a library, implementing input and output functions.\\

All these libraries were built in order to implement a radix sort solution.\\

In the \hyperref[chapter:Objectives]{first chapter}, we will talk about the objectives of this project, and the problems we encountered.\\
In the \hyperref[chapter:Algorithms]{second chapter}, we will introduce the founded solutions, with their algorithms.
Finally, in the \hyperref[chapter:IO-Lib]{third chapter}, we will quickly talk about  the io library

\tableofcontents{}

\chapter{Objectives and problem statements} \label{chapter:Objectives}
The main objective of this project was to implement two libraries, with one that could perform a Radix Sort, together with an user-friendly executable to tests the library.\\
Radix sort is a algorithm used to sort numbers. It works as the following (with m the maximum number of digit in the elements of the list, b the base of the numbers)\footnote{Note that this is a Least Significant Digit (LSD) Radix Sort variant} :\\
\begin{itemize}
	\item[1]{} let \(n\) equals 1, and make a ListOfList of B lists
	\item[2]{} Traverse all the list, watching for the \(n^{th}\) digit of each value, starting from the right.
	\item[3]{} Store each element in its corresponding List. If the watched digit was 1, it should go in the  list 1.
	\item[4]{} Convert the list of list into a single list, enqueing them.
	\item[5]{} if m is greater or equal to n, increase n by 1 and step back to 2.
\end{itemize}

In order to complete this objective, we had to face several problems :
\begin{itemize}
	\item{} Implement functions to manage BaseNIntegerList
	\item{} Implement functions to manage Bucket List
	\item{} Implement functions to have a user-friendly interface
\end{itemize}

Besides being fully working, all these functions should be efficient and without memory leaks.

\chapter{Algorithms} \label{chapter:Algorithms}
	\section{BaseNIntegerList}
	The \textit{BaseNIntegerList} library implements a doubly linked list structure, with management functions as well.\\
	A BaseNIntegerList contains a pointer to its first and last element, an integer telling its base, and an integer telling the number of elements within it.\\
	An Element contains a value (stored as an array of characters in the base of the list), a pointer to its predecessor, and a pointer to its successor.\\

		\begin{minipage}{\linewidth}
			\textit{createIntegerList}: \textit{Integer} $\rightarrow$ \textit{BaseNIntegerList}\\
			Creates a new BaseNIntegerList for storing integers in the specified base.
			\label{algo:BNIL-CreateIntegerList}
			\begin{lstlisting}[breaklines]
function CreateIntegerList (base: integer): BaseNIntegerList
Begin
|       l: BaseNIntegerList
|       head(l) <- undefined
|       tail(l) <- undefined
|       base(l) <- base
|       size(l) <- 0
|
|       Createintegerlist <- l
Done
\end{lstlisting}

		\end{minipage}
		\nxtalgo{}

		\begin{minipage}{\linewidth}
			\textit{isEmpty}: \textit{BaseNIntegerList} $\rightarrow$ \textit{Boolean}\\
			Returns true if the specified list is empty, false otherwise.
			\label{algo:BNIL-IsEmpty}
			\begin{lstlisting}[breaklines]
function IsEmpty (l: BaseNIntegerList): Boolean
Begin
|    if (size(l) = 0)
|    then
|    |    IsEmpty <- true
|    else
|    |    IsEmpty <- false
|    endif
Done
\end{lstlisting}

		\end{minipage}
		\nxtalgo{}

		\begin{minipage}{\linewidth}
			\textit{insertHead}: \textit{BaseNIntegerList} $\times$ \textit{char*} $\rightarrow$ \textit{BaseNIntegerList}\\
			Adds the specified integer (char*, represented in the considered base) at the beginning of the specified list.\\
			\ov
			\begin{itemize}
				\item Creates a new element
				\item Roots its following element to the actual head of the list
				\item Reroots the head of the list to it
				\item Increases the size of the list
			\end{itemize}
			\label{algo:BNIL-InsertHead}
			\begin{lstlisting}[breaklines]
function InsertHead (l: BaseNIntegerList, v: array<characters>): BaseNIntegerList
Begin
|    newel: ListElem*
|    newel <- alloc(ListElem)
|    value(newel) <- v
|    next(newel) <- head(l)
|    previous(newel) <- undefined
|
|    if (not IsEmpty(l))
|    then
|    |    previous(head(l)) <- newel
|    else
|    |    tail(l) <- newel
|    endif
|
|    head(l) <- newel
|    size(l) <- size(l) + 1
|
|    InsertHead <- l
Done
\end{lstlisting}

		\end{minipage}
		\nxtalgo{}

		\begin{minipage}{\linewidth}
			\textit{insertTail}: \textit{BaseNIntegerList} $\times$ \textit{char*} $\rightarrow$ \textit{BaseNIntegerList}\\
			Adds the specified integer (char*) at the end of the specified list.\\
			\ov
			\begin{itemize}
				\item Creates a new element
				\item Roots its previous element to the actual tail of the list
				\item Reroots the tail of the list to it
				\item Increases the size of the list
			\end{itemize}
			\label{algo:BNIL-InsertTail}
			\begin{lstlisting}[breaklines]
function InsertTail (l: BaseNIntegerList, v: array<characters>): BaseNIntegerList
Begin
|    newel: ListElem*
|    newel <- alloc(ListElem)
|    value(newel) <- v
|    next(newel) <- NULL
|    previous(newel) <- tail(l)
|
|    if (not IsEmpty(l))
|    then
|    |    next(tail(l)) <- newel
|    else
|    |    head(l) <- newel
|    endif
|
|    tail(l) <- newel
|    size(l) <- size(l) + 1
|
|    InsertTail <- l
Done
\end{lstlisting}

		\end{minipage}
		\nxtalgo{}

		\begin{minipage}{\linewidth}
			\textit{removeHead}: \textit{BaseNIntegerList} $\times$ \textit{char*} $\rightarrow$ \textit{BaseNIntegerList}\\
			Removes the first element of the specified list.\\
			\ov \textit{ (assuming the list has more than one element)}
			\begin{itemize}
				\item Reroots the head of the list to the second element
				\item Deletes the new head's previous element
				\item Reroots the head's previous element to nothing
				\item Decreases the size of the list
			\end{itemize}
			\label{algo:BNIL-RemoveHead}
			\begin{lstlisting}[breaklines]
function RemoveHead(l: BaseNIntegerList): BaseNIntegerList
Begin
|    if (not IsEmpty(l))
|    then
|    |    if (size(l) = 1)
|    |    then
|    |    |    free(value(head(l)))
|    |    |    free(head(l))
|    |    |    head(l) <- NULL
|    |    |    tail(l) <- NULL
|    |    |    size(l) <- 0
|    |    else
|    |    |    head(l) <- next(head(l))
|    |    |    free(value(previous(head(l))))
|    |    |    free(previous(head(l)))
|    |    |    previous(head(l)) <- NULL
|    |    |    size(l) <- size(l) - 1
|    |    endif
|    endif
done
\end{lstlisting}

		\end{minipage}
		\nxtalgo{}

		\begin{minipage}{\linewidth}
			\textit{removeTail}: \textit{BaseNIntegerList} $\times$ \textit{char*} $\rightarrow$ \textit{BaseNIntegerList}\\
			Removes the last element of the specified list.\\
			\ov \textit{ (assuming the list has more than one element)}
			\begin{itemize}
				\item Reroots the tail of the list to the second-to-last element
				\item Deletes the new tail's next element
				\item Reroots the tail's next element to nothing
				\item Decreases the size of the list
			\end{itemize}
			\label{algo:BNIL-RemoveTail}
			\begin{lstlisting}[breaklines]
function RemoveTail(l: BaseNIntegerList): BaseNIntegerList
Begin
|    if (not IsEmpty(l))
|    then
|    |    if (size(l) = 1)
|    |    then
|    |    |    free(value(head(l)))
|    |    |    free(head(l))
|    |    |    head(l) <- undefined
|    |    |    tail(l) <- undefined
|    |    |    size(l) <- 0
|    |    else
|    |    |    tail(l) <- previous(tail(l))
|    |    |    free(value(next(tail(l))))
|    |    |    free(next(tail(l)))
|    |    |    next(tail(l)) <- undefined
|    |    |    size(l) <- size(l) - 1
|    |    endif
|    endif
|
|    RemoveTail <- l
Done
\end{lstlisting}

		\end{minipage}
		\nxtalgo{}

		\begin{minipage}{\linewidth}
			\textit{deleteIntegerList}: \textit{BaseNIntegerList} $\rightarrow \emptyset$\\
			Clears and deletes the specified BaseNIntegerList (free previously allocated memory).\\
			\label{algo:BNIL-DeleteIntegerList}
			\begin{lstlisting}[breaklines]
procedure DeleteIntegerList (l: BaseNIntegerList*)
Begin
|    while (not IsEmpty(*l)) do
|    |    *l <- RemoveHead(*l)
|    done
Done
\end{lstlisting}

		\end{minipage}
		\nxtalgo{}

		\begin{minipage}{\linewidth}
			\textit{sumIntegerList}: \textit{BaseNIntegerList} $\rightarrow$ \textit{char*}\\
			Sums all the intgers defined in the specified list using the Binary addition (base 2) and returns the corresponding results as an integer (char*) defined in the base of the list.\\
			\ov\\
			This function traverses the whole list, gradually summing each element.
			\label{algo:BNIL-SumIntegerList}
			\begin{lstlisting}[breaklines]
function SumIntegerList(l: BaseNIntegerList): array<characters>
Begin
|    if (not IsEmpty(l))
|    then
|    |    element: ListElem*
|    |    element <- head(l)
|    |
|    |    if (size(l) = 1)
|    |    then
|    |    |    SumIntegerList <- copy(value(element))
|    |    else
|    |    |    s: array<characters>
|    |    |    tmp: array<characters>
|    |    |    s <- value(element)
|    |    |    element <- next(element)
|    |    |
|    |    |    do
|    |    |    |    s <- SumBase(s, element->value, l.base)
|    |    |    |    free(tmp)
|    |    |    |    tmp <- s
|    |    |    |    element <- next(element)
|    |    |    while (element ≠ NULL)
|    |    |
|    |    |    Sumintegerlist <- s
|    |    endif
|    else
|    |    SumIntegerList <- NULL
|    endif
done
\end{lstlisting}

			\textit{SumBase(a,b,c)} returns the sum $(a+b)$, with $a$ and $b$ in base $c$
		\end{minipage}
		\nxtalgo{}

		\begin{minipage}{\linewidth}
			\textit{BaseToInt}: \textit{char*} $\times$ \textit{integer} $\rightarrow$ \textit{integer}\\
			Converts the value char* (in the given base) into integer.
			\ov\\
			\begin{itemize}
				\item 
			\end{itemize}
			\label{algo:BNIL-BaseToInt}
			nb : GetValue returns the value represented by a character. (GetValue('F') = 16)
\begin{lstlisting}
function BaseToInt (v: array<characters>, base: positive integer): positive integer
Begin
|    n, temp, size, i: positive integers
|    n <- 0
|    temp <- 1
|    size <- arraySize(v)
|
|    if (size > 0)
|    then
|    |    n <- GetValue(v[0])
|    |
|    |    for i from 1 to (size - 1) do
|    |    |    temp <- base * temp
|    |    |    n <- n + GetValue(v[0]) * temp
|    |    done
|    |
|    endif
|
|    BaseToInt <- n
Done
\end{lstlisting}

			\textit{GetValue(a)} returns the value represented by the character $a$
			(ie : \textit{(GetValue('F')} $= 16$)
		\end{minipage}
		\nxtalgo{}

		\begin{minipage}{\linewidth}
			\textit{IntToBase}: \textit{integer} $\times$ \textit{integer} $\rightarrow$ \textit{char*}\\
			Converts the integer value into char* in the given base.
			\label{algo:BNIL-IntToBase}
			\begin{lstlisting}[breaklines]
function IntToBase(v: positive integer, base: positive integer): array<characters>
Begin
|    k, i: positive integers
|    w, base_digit: array<characters>
|
|    k <- 1
|    i <- base
|    base_digit <- {'0','1', ..., '9', 'A', 'B', ..., 'Z'}
|    
|    while (v >= i) do
|    |    i <- i*base
|    |    k <- k + 1
|    done
|
|    w <- alloc(k*characters)
|    w[0] <- '0'
|    k <- 0
|
|    while (v > 0) do
|    |    w[k] <- base_digit[v%base]
|    |    k <- k + 1
|    |    v <- v/base
|    done
|    IntToBase <- w
Done
\end{lstlisting}

		\end{minipage}
		\nxtalgo{}

		\begin{minipage}{\linewidth}
			\textit{ConvertBaseToBinary}: \textit{char*} $\times$ \textit{integer} $\rightarrow$ \textit{char*}\\
			Converts the specied integer (char*) represented with the specified base (Integer, second parameter) into a corresponding binary integer (base 2).
			\label{algo:BNIL-ConvertBaseToBinary}
			\begin{lstlisting}[breaklines]
function ConvertBaseToBinary (v: array<characters>, base: positive integer): array<characters>
Begin
|    ConvertBaseToBinary <- IntToBase(BaseToInt(v, base), 2)
Done
\end{lstlisting}

		\end{minipage}
		\nxtalgo{}

		\begin{minipage}{\linewidth}
			\textit{ConvertBinaryToBase}: \textit{char*} $\times$ \textit{integer} $\rightarrow$ \textit{char*}\\
			Converts an integer represented using a binary base (base 2) into a corresponding integer represented with the specified base (Integer, second parameter).
			\label{algo:BNIL-ConvertBinaryToBase}
			\begin{lstlisting}[breaklines]
function ConvertBinaryToBase (v: array<characters>, base: positive integer): array<characters>
Begin
|    ConvertBinaryToBase <- IntToBase(BaseToInt(v, 2), Base)
Done
\end{lstlisting}

		\end{minipage}
		\nxtalgo{}

		\begin{minipage}{\linewidth}
			\textit{SumBase}: \textit{char*} $\times$  \textit{char*} $\times$  \textit{integer} $\rightarrow$ \textit{char*}\\
			Sum two integer (char*) represented with the specified base (Integer, third parameter) and return the sum in the same base.
			\label{algo:BNIL-SumBase}
			\begin{lstlisting}[breaklines]
function SumBase(a: array<characters>, b: array<characters>, base: positive integer): array<characters>
Begin
|    i, j, k, a_len, b_len, remainder: integers
|    s, base_digits : array<characters>
|
|    i <- 0
|    j <- 0
|    k <- 0
|    a_len <- arraySize(a)
|    b_len <- arraySize(b)
|    base_digits <- { '0', '1', \ldots, '9', 'A', 'B', \ldots, 'Z' }
|
|    while ((i < a_len) and (j < b_len)) do
|    |    remainder <- remainder + GetValue(a[i]) + GetValue(b[j])
|    |    s[k] <- base_digits[remainder\%base]
|    |    remainder <- remainder/base
|    |    k <- k + 1
|    |    i <- i + 1
|    |    j <- j + 1
|    done
|
|    while (i < a_len) do
|    |    remainder <- remainder + Getvalue(a[i])
|    |    s[k] <- base_digits[remainder\%base]
|    |    remainder <- remainder/base
|    |    k <- k+1
|    |    i <- i+1
|    done
|    
|    while (j < b_len) do
|    |    remainder <- remainder + Getvalue(a[i])
|    |    s[k] <- base_digits[remainder\%base]
|    |    remainder <- remainder/base
|    |    k <- k+1
|    |    j <- j+1
|    done
|
|    if (remainder != 0)
|    then
|    |    s[k] <- base_digits[remainder]
|    endif
|
|    SumBase <- s
Done
\end{lstlisting}
		\end{minipage}

		\begin{minipage}{\linewidth}
			\textit{SumBase}: \textit{char*} $\times$  \textit{char*} $\times$  \textit{integer} $\rightarrow$ \textit{char*}\\
			Sum two integer (char*) represented in binary base and return the sum in binary base.
			\label{algo:BNIL-SumBinary}\\
			\textbf{Similar to \textit{SumBase}}
		\end{minipage}

	\section{BaseNIntegerListOfList}
		The \textit{BaseNIntegerListOfList} library implements a structure, with management functions as well.\\
		The BaseNIntegerListOfList structure is basically an array of n lists in base n.\\

		\begin{minipage}{\linewidth}
			\textit{createBucketList}: \textit{Integer} $\rightarrow$ \textit{BaseNIntegerListOfList}\\
			Creates a BaseNIntegerListOfList for storing list of integers in base N (N being the specified integer, first parameter).
			\label{algo:BNIL-CreateBucketList}
			\begin{lstlisting}[breaklines]
function CreateBucketList (N: positive integer): BaseNIntegerListOfList
Begin
|    l: BaseNIntegerListOfList
|
|    base(l) <- N
|    list(l) <- array<BaseNIntegerList>[N] //From 0 to N-1
|
|    for i from 0 to (N - 1) do
|    |    list(l)[i] <- CreateIntegerList(N)
|    done
|
|    CreateBucketList <- l
Done
\end{lstlisting}

		\end{minipage}
		\nxtalgo{}

		\begin{minipage}{\linewidth}
			\textit{buildBucketList}: \textit{BaseNIntegerList} $\times$ \textit{Integer} $\rightarrow$ \textit{BaseNIntegerListOfList}\\
			Builds a new BaseNIntegerListOfList according to the specified BaseNIntegerList and considering the specified digit position (rightmost).
			\label{algo:BNIL-BuildBucketList}
			\begin{lstlisting}[breaklines]
function BuildBucketList (list: BaseNIntegerList, digit_pos: Integer): BaseNIntegerListOfList
Begin
|    list_of_list: BaseNIntegerListOfList
|    number: array<characters>
|
|    list_of_list <- CreateBucketList(base(list))
|    list_element <- head(list)
|
|    while (list_element != undefined) do
|    |    number <- copy(value(list_element))
|    |
|    |    if (arraySize(number) > digit_pos)
|    |    then
|    |    |    list_of_list <- AddIntegerIntoBucket (list_of_list, number, GetValue(number[digit_pos]))
|    |    else
|    |    |    list_of_list <- AddIntegerIntoBucket (list_of_list, number, 0)
|    |    endif
|    |
|    |    list_of_list <- next(list_of_list)
|    done
|
|    Buildbucketlist <- list_of_list
|
Done
\end{lstlisting}

		\end{minipage}
		\nxtalgo{}

		\begin{minipage}{\linewidth}
			\textit{buildIntegerList}: \textit{BaseNIntegerListOfList} $\rightarrow$ \textit{BaseNIntegerList}\\
			Builds a new BaseNIntegerList from the specified BaseNIntegerListOfList (respecting the ascending order on the buckets).
			\label{algo:BNIL-BuildIntegerList}
			\begin{lstlisting}[breaklines]
function BuildIntegerList (list_of_list: BaseNIntegerListOfList): BaseNIntegerList
Begin
|    output: BaseNIntegerList
|    val: array<characters>
|    inserted_elem: ListElem*
|
|    output <- CreateIntegerList(base(list_of_list)
|
|    for i from 0 to (base(list_of_list) - 1) do
|    |
|    |    inserted_elem <- head(list_of_list[i])
|    |
|    |    while (inserted_elem ≠ NULL) do
|    |    |
|    |    |    value <- copy(value(inserted_elem))
|    |    |    output <- InsertTail(output, value)
|    |    |    inserted_elem <- next(inserted_elem)
|    |    done
|    done
|
|    BuildIntegerList <- output
Done
\end{lstlisting}

		\end{minipage}
		\nxtalgo{}

		\begin{minipage}{\linewidth}
			\textit{addIntegerIntoBucket}: \textit{BaseNIntegerListOfList} $\times$  \textit{char*} $\times$  \textit{integer} $\rightarrow$ \textit{BaseNIntegerListOfList}\\
			Adds a new integer (char*) at the end of the specified list in bucket N (N being the third parameter).
			\label{algo:BNIL-AddIntegerIntoBucket}
			\begin{lstlisting}[breaklines]
function AddIntegerIntoBucket (list_of_list: BaseNIntegerListOfList, v: array<characters>, bucket_number: integer): BaseNIntegerListOfList
Begin
|    list(list_of_list)[bucket_number] <- InsertTail(list(list_of_list)[bucket_number], v)
|    Addintegerintobucket <- list_of_list
Done
\end{lstlisting}

		\end{minipage}
		\nxtalgo{}

		\begin{minipage}{\linewidth}
			\textit{deleteBucketList}: \textit{BaseNIntegerListOfList} $\rightarrow \emptyset$\\
			Clears and deletes the specified BaseNIntegerListOfList (free the previously allocated memory).
			\label{algo:BNIL-DeleteBucketList}
			\begin{lstlisting}[breaklines]
procedure DeleteBucketList (list_of_list: BaseNIntegerListOfList)
Begin
|    for i from 0 to (base(list_of_list) - 1) do
|    |    DeleteIntegerList(list(list_of_list)[i])
|    done
|    free (list(list_of_list))
|    list(list_of_list) <- NULL
Done
\end{lstlisting}

		\end{minipage}
		\nxtalgo{}

		\begin{minipage}{\linewidth}
			\textit{radixsort}: \textit{BaseNIntegerListOfList} $\rightarrow \emptyset$\\
			Sorts the specified BaseNIntegerList using the proposed radix sort approach.
			\label{algo:BNIL-RadixSort}
			\begin{lstlisting}[breaklines]
function RadixSort (list: BaseNIntegerList): BaseNIntegerList
Begin
|
|    if (IsEmpty(list))
|    then
|    |    RadixSort <- CreateIntegerList(base(list))
|    else
|    |    max_size, tmp: integers
|    |    elem: ListElem*
|    |    list_of_list: BaseNIntegerListOfList
|    |
|    |    max_size <- arraySize(value(head(list)))
|    |    elem <- head(list)
|    |
|    |    while (next(elem) != undefined) do
|    |    |    elem <- next(elem)
|    |    |    tmp <- arraySize(value(elem))
|    |    |    if (tmp > max_size)
|    |    |    then
|    |    |    |    max_size <- tmp
|    |    |    endif
|    |    done
|    |
|    |    list_of_list <- BuildBucketList(list, 0)
|    |    sorted_list: <- BuildIntegerList(List_of_list)
|    |    DeleteBucketList(list_of_list)
|    |
|    |    for i from 1 to (max_size - 1) do
|    |    |
|    |    |    list_of_list <- BuildBucketList(sorted_list, i)
|    |    |    DeleteIntegerList(sorted_list)
|    |    |    sorted_list <- BuildIntegerList(list_of_list)
|    |    |    DeleteBucketList(list_of_list)
|    |    done
|    |
|    |    RadixSort <- sorted_list
|    endif
Done
\end{lstlisting}

		\end{minipage}
		\nxtalgo{}

\chapter{Input Output Library} \label{chapter:IO-Lib}
	The \textit{io} library is a library that implements some useful input and output functions, especially for BaseNIntegerList.\\
	We used it to provide a nice CLI for the user.\\
	\section{Usefull functions}
		\textit{Clear}\\
			\qquad Clears the screen\\
		\textit{SetTextColor, SetTextAttributes, SetBgColor}\\
			\qquad Changes the colors and the attributes of the text (bold,\ldots)\\
		\textit{PrintList}\\
			\qquad Prints a list. All numbers displayed are between brackets, and a whitespace is displayed every
			3 or 4 digits (depending on the base)\\
		\textit{Menu}\\
			\qquad Prints a menu where you can navigate using arrows and page up/down, and select your choice
			pressing enter.\\
		\textit{GetNumberWithinRange}\\
			\qquad Waits for the user to input a number within a given range. If the user inputs a number out of
			the asked range, the input is voided and to be repeated\\
		\textit{GetList}\\
			\qquad Asks the user to input a list. It can be manually inputed or randomly generated.\\
		\textit{InstantGetChar}\\
			\qquad Similar to a normal getChar() but it doesn't wait for the user to press enter.\\
	\section{Menu}
		In this section, we will talk a little bit more in detail about the \textit{Menu} and the \textit{GetList} functions.\\
		\newline
		The \textit{Menu} function took us quite some times. Its prototype is the following :\\
		\textit{function Menu}(choices: \textit{array<characters>}, nb_choices: \textit{integer}, text_color: \textit{array<characters>}, bg_color: \textit{array<characters}, choice: \textit{integer}): \textit{integer}
		\begin{itemize}
			\item{} choices is a string with each choice, separated by a null character
			\item{} nb_choices is the number of choices the user has
			\item{} text_color is the color to use for unselected choice display
			\item{} bg_color is the same for the background color
			\item{} choice is the default choice
		\end{itemize}
		The selected choice will be displayed in inverted color (background color becomes the text color and vice-versa).\\
		All choices are display in an rectangle, which color is 'bg\_color', with two grey lines on the top and right.\\
		\newline
		\ov
		The first step is to convert the choices array into sub arrays containing each one a single choice. This is done by scanning for null characters.\\
		After that, we will look for the longest choice, to know the width of our rectangle (the height is given by the number of choices).\\
		When this is done, we will compute the coordinate of the left character of each choice. This is not really a required step for the way we used the menu, but thanks to that, it can be easily adapted to display choices with a centered or right aligned display of each choice.\\
		The hardest job is done. All what's left is to wait for user to press a 'choice modifier key' -- in that case, change the selection -- or enter -- in this case, return.\\

	
\chapter{Conclusion} \label{chapter:Conclusion}
	Maxime Pinard worked more on the functional part of the the project while Lucas Lazare focused more on the interface part and the makefile
	
	Base manipulation functions were very interesting to think about and the limitation of number convertion to 2 147 483 647 can be overcome by implementing a conversion function using a function that compute the multiplication in any base, this function would use the addition in any base already in the lib.

	We apreciated to implement the CLI because unlike the Lists libraries the result is visual and user adjustable and colorful. 

	If we had to change someting in this programme this would be to use more pointers.
	Use pointers instead of copy the numbers (char*) and even just reroot the ListElem* of the lists into the bucketlists and after reroot them in the list. With this modification we can use only one ListOfList and one List for the RadixSort.
\end{document}
